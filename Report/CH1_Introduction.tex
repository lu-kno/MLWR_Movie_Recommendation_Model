\section{Introduction} 
\IEEEPARstart{T}{he} abundance of digital media today has turned the task of selecting a film to watch into a daunting exercise. Amidst this sea of choices, personalized movie recommendation systems provide a much-needed compass, guiding users towards films that align with their tastes and preferences. As a response to this growing demand, our project leverages machine learning, with a particular emphasis on clustering, to construct a robust movie recommendation model.

The Movie Lens Database (ML) serves as a comprehensive archive of movie metadata, covering films spanning different periods and genres along with ratings and tags given by individual users. This project focuses on utilizing this dataset, specifically the user-generated tags, to develop a predictive model. The genome scores provide a value that indicated the degree of relevance that exists between each tag and movie. These user tags encapsulate various aspects of the film, such as its genre, themes, narrative devices, and unique identifiers. While these tags are insightful, due to the human nature by which these tags were created, their sheer volume and diversity pose a challenge to the development of a manageable and efficient prediction model.

{\color{red}vvv I'll be back vvv}\\
In order to solve the prediction problem the k-means clustering algorithm will be utilized. By interpreting the profuse tag space geometrically, k-means clustering presents an intuitive and efficient solution to the movie recommendation problem. This approach will be utilized to form 'genre spaces' or clusters, predicated on tag similarity, thereby laying the groundwork for movie predictions.

The goal of this project is to create a model for movie recommendations based on a user's previous watching history. The process involves preprocessing the ML database tags to reduce feature size, and subsequently applying k-means clustering. With this model, we aspire to provide effective guidance to users across the expansive cinematic landscape
