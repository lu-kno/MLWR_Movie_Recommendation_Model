\section{Model implementation for Recommendation Service}

In order to provide a recommendation, an instance of a user need to be created in order to analyze the user's preferences. This instance is created by selecting a user from the database and extracting the list of movies that the user has seen along with the rating that the user gave to each movie. The rating is normalized between 0 and 1.

\subsection{Characterization of the users' preference vector}

The available data from a given User requires further processing in order to be used for the recommendation algorithm. The first step is to define the clusters to which each of the films belongs to, which is performed using the model described in the previous section.

The movies where grouped into the different clusters and the sum of ratings for each group was calculated. This provided a measure of the rating a user would most likely give to a movie belonging to a certain cluster relative to the other clusters. These values were then normalized so their sum was equal to 1. This resulted in a vector of length equal to the number of clusters representing the individual preferences of each user for each cluster based on their past behavior.

This method allows the characterization of each user in terms of the relevance of each cluster for each user. With the resulting data, it can be assumed that the user's preferences and the movies' characteristics are represented by the same vector space.

With this information, it can be assumed that the movies a user would like to see the most are the ones that belong to the cluster that the user has the highest preference for, followed by the movies that belong to the cluster that the user has the second highest preference for, and so on.

\subsection{Recommendation algorithm}

The recommendation algorithm is based on the assumption that the user's current preferences have not changed too greatly since the last time the user rated a movie. This assumption is reasonable since the user's preferences are based on the user's behavior over time.

This process is implemented as follows:

The first step is to define the cluster from which to take the movie recommendation from using a random number generator with the user's preference vector as the probability distribution
This is followed by selecting a movie from the cluster that the user has not seen yet.% with a preference for the movies with the highest rating

If the user has seen all the movies from the cluster, then a new random cluster is selected and the process is repeated.

